% Options for packages loaded elsewhere
\PassOptionsToPackage{unicode}{hyperref}
\PassOptionsToPackage{hyphens}{url}
\PassOptionsToPackage{dvipsnames,svgnames,x11names}{xcolor}
%
\documentclass[
  letterpaper,
  DIV=11,
  numbers=noendperiod]{scrartcl}

\usepackage{amsmath,amssymb}
\usepackage{iftex}
\ifPDFTeX
  \usepackage[T1]{fontenc}
  \usepackage[utf8]{inputenc}
  \usepackage{textcomp} % provide euro and other symbols
\else % if luatex or xetex
  \usepackage{unicode-math}
  \defaultfontfeatures{Scale=MatchLowercase}
  \defaultfontfeatures[\rmfamily]{Ligatures=TeX,Scale=1}
\fi
\usepackage{lmodern}
\ifPDFTeX\else  
    % xetex/luatex font selection
\fi
% Use upquote if available, for straight quotes in verbatim environments
\IfFileExists{upquote.sty}{\usepackage{upquote}}{}
\IfFileExists{microtype.sty}{% use microtype if available
  \usepackage[]{microtype}
  \UseMicrotypeSet[protrusion]{basicmath} % disable protrusion for tt fonts
}{}
\makeatletter
\@ifundefined{KOMAClassName}{% if non-KOMA class
  \IfFileExists{parskip.sty}{%
    \usepackage{parskip}
  }{% else
    \setlength{\parindent}{0pt}
    \setlength{\parskip}{6pt plus 2pt minus 1pt}}
}{% if KOMA class
  \KOMAoptions{parskip=half}}
\makeatother
\usepackage{xcolor}
\setlength{\emergencystretch}{3em} % prevent overfull lines
\setcounter{secnumdepth}{5}
% Make \paragraph and \subparagraph free-standing
\ifx\paragraph\undefined\else
  \let\oldparagraph\paragraph
  \renewcommand{\paragraph}[1]{\oldparagraph{#1}\mbox{}}
\fi
\ifx\subparagraph\undefined\else
  \let\oldsubparagraph\subparagraph
  \renewcommand{\subparagraph}[1]{\oldsubparagraph{#1}\mbox{}}
\fi

\usepackage{color}
\usepackage{fancyvrb}
\newcommand{\VerbBar}{|}
\newcommand{\VERB}{\Verb[commandchars=\\\{\}]}
\DefineVerbatimEnvironment{Highlighting}{Verbatim}{commandchars=\\\{\}}
% Add ',fontsize=\small' for more characters per line
\usepackage{framed}
\definecolor{shadecolor}{RGB}{241,243,245}
\newenvironment{Shaded}{\begin{snugshade}}{\end{snugshade}}
\newcommand{\AlertTok}[1]{\textcolor[rgb]{0.68,0.00,0.00}{#1}}
\newcommand{\AnnotationTok}[1]{\textcolor[rgb]{0.37,0.37,0.37}{#1}}
\newcommand{\AttributeTok}[1]{\textcolor[rgb]{0.40,0.45,0.13}{#1}}
\newcommand{\BaseNTok}[1]{\textcolor[rgb]{0.68,0.00,0.00}{#1}}
\newcommand{\BuiltInTok}[1]{\textcolor[rgb]{0.00,0.23,0.31}{#1}}
\newcommand{\CharTok}[1]{\textcolor[rgb]{0.13,0.47,0.30}{#1}}
\newcommand{\CommentTok}[1]{\textcolor[rgb]{0.37,0.37,0.37}{#1}}
\newcommand{\CommentVarTok}[1]{\textcolor[rgb]{0.37,0.37,0.37}{\textit{#1}}}
\newcommand{\ConstantTok}[1]{\textcolor[rgb]{0.56,0.35,0.01}{#1}}
\newcommand{\ControlFlowTok}[1]{\textcolor[rgb]{0.00,0.23,0.31}{#1}}
\newcommand{\DataTypeTok}[1]{\textcolor[rgb]{0.68,0.00,0.00}{#1}}
\newcommand{\DecValTok}[1]{\textcolor[rgb]{0.68,0.00,0.00}{#1}}
\newcommand{\DocumentationTok}[1]{\textcolor[rgb]{0.37,0.37,0.37}{\textit{#1}}}
\newcommand{\ErrorTok}[1]{\textcolor[rgb]{0.68,0.00,0.00}{#1}}
\newcommand{\ExtensionTok}[1]{\textcolor[rgb]{0.00,0.23,0.31}{#1}}
\newcommand{\FloatTok}[1]{\textcolor[rgb]{0.68,0.00,0.00}{#1}}
\newcommand{\FunctionTok}[1]{\textcolor[rgb]{0.28,0.35,0.67}{#1}}
\newcommand{\ImportTok}[1]{\textcolor[rgb]{0.00,0.46,0.62}{#1}}
\newcommand{\InformationTok}[1]{\textcolor[rgb]{0.37,0.37,0.37}{#1}}
\newcommand{\KeywordTok}[1]{\textcolor[rgb]{0.00,0.23,0.31}{#1}}
\newcommand{\NormalTok}[1]{\textcolor[rgb]{0.00,0.23,0.31}{#1}}
\newcommand{\OperatorTok}[1]{\textcolor[rgb]{0.37,0.37,0.37}{#1}}
\newcommand{\OtherTok}[1]{\textcolor[rgb]{0.00,0.23,0.31}{#1}}
\newcommand{\PreprocessorTok}[1]{\textcolor[rgb]{0.68,0.00,0.00}{#1}}
\newcommand{\RegionMarkerTok}[1]{\textcolor[rgb]{0.00,0.23,0.31}{#1}}
\newcommand{\SpecialCharTok}[1]{\textcolor[rgb]{0.37,0.37,0.37}{#1}}
\newcommand{\SpecialStringTok}[1]{\textcolor[rgb]{0.13,0.47,0.30}{#1}}
\newcommand{\StringTok}[1]{\textcolor[rgb]{0.13,0.47,0.30}{#1}}
\newcommand{\VariableTok}[1]{\textcolor[rgb]{0.07,0.07,0.07}{#1}}
\newcommand{\VerbatimStringTok}[1]{\textcolor[rgb]{0.13,0.47,0.30}{#1}}
\newcommand{\WarningTok}[1]{\textcolor[rgb]{0.37,0.37,0.37}{\textit{#1}}}

\providecommand{\tightlist}{%
  \setlength{\itemsep}{0pt}\setlength{\parskip}{0pt}}\usepackage{longtable,booktabs,array}
\usepackage{calc} % for calculating minipage widths
% Correct order of tables after \paragraph or \subparagraph
\usepackage{etoolbox}
\makeatletter
\patchcmd\longtable{\par}{\if@noskipsec\mbox{}\fi\par}{}{}
\makeatother
% Allow footnotes in longtable head/foot
\IfFileExists{footnotehyper.sty}{\usepackage{footnotehyper}}{\usepackage{footnote}}
\makesavenoteenv{longtable}
\usepackage{graphicx}
\makeatletter
\def\maxwidth{\ifdim\Gin@nat@width>\linewidth\linewidth\else\Gin@nat@width\fi}
\def\maxheight{\ifdim\Gin@nat@height>\textheight\textheight\else\Gin@nat@height\fi}
\makeatother
% Scale images if necessary, so that they will not overflow the page
% margins by default, and it is still possible to overwrite the defaults
% using explicit options in \includegraphics[width, height, ...]{}
\setkeys{Gin}{width=\maxwidth,height=\maxheight,keepaspectratio}
% Set default figure placement to htbp
\makeatletter
\def\fps@figure{htbp}
\makeatother

\usepackage{codehigh}
\usepackage{float}
\usepackage{tabularray}
\UseTblrLibrary{booktabs}
\NewTableCommand{\tinytableDefineColor}[3]{\definecolor{#1}{#2}{#3}}
\KOMAoption{captions}{tableheading}
\makeatletter
\@ifpackageloaded{caption}{}{\usepackage{caption}}
\AtBeginDocument{%
\ifdefined\contentsname
  \renewcommand*\contentsname{Table of contents}
\else
  \newcommand\contentsname{Table of contents}
\fi
\ifdefined\listfigurename
  \renewcommand*\listfigurename{List of Figures}
\else
  \newcommand\listfigurename{List of Figures}
\fi
\ifdefined\listtablename
  \renewcommand*\listtablename{List of Tables}
\else
  \newcommand\listtablename{List of Tables}
\fi
\ifdefined\figurename
  \renewcommand*\figurename{Figure}
\else
  \newcommand\figurename{Figure}
\fi
\ifdefined\tablename
  \renewcommand*\tablename{Table}
\else
  \newcommand\tablename{Table}
\fi
}
\@ifpackageloaded{float}{}{\usepackage{float}}
\floatstyle{ruled}
\@ifundefined{c@chapter}{\newfloat{codelisting}{h}{lop}}{\newfloat{codelisting}{h}{lop}[chapter]}
\floatname{codelisting}{Listing}
\newcommand*\listoflistings{\listof{codelisting}{List of Listings}}
\makeatother
\makeatletter
\makeatother
\makeatletter
\@ifpackageloaded{caption}{}{\usepackage{caption}}
\@ifpackageloaded{subcaption}{}{\usepackage{subcaption}}
\makeatother
\ifLuaTeX
  \usepackage{selnolig}  % disable illegal ligatures
\fi
\usepackage{bookmark}

\IfFileExists{xurl.sty}{\usepackage{xurl}}{} % add URL line breaks if available
\urlstyle{same} % disable monospaced font for URLs
\hypersetup{
  pdftitle={tinytable},
  colorlinks=true,
  linkcolor={blue},
  filecolor={Maroon},
  citecolor={Blue},
  urlcolor={Blue},
  pdfcreator={LaTeX via pandoc}}

\title{\texttt{tinytable}}
\author{}
\date{}

\begin{document}
\maketitle

\renewcommand*\contentsname{Table of contents}
{
\hypersetup{linkcolor=}
\setcounter{tocdepth}{3}
\tableofcontents
}
\clearpage

\texttt{tinytable} is a small but powerful \texttt{R} package to draw
HTML, LaTeX, PDF, Markdown, and Typst tables. The interface is
minimalist, but it gives users direct and convenient access to powerful
frameworks to create endlessly customizable tables.

This tutorial introduces the main functions of the package. It is
available in two versions:

\begin{itemize}
\tightlist
\item
  \href{tutorial.pdf}{PDF}
\item
  \href{tutorial.html}{HTML}
\end{itemize}

\section{Tiny Tables}\label{tiny-tables}

\begin{Shaded}
\begin{Highlighting}[]
\FunctionTok{library}\NormalTok{(tinytable)}
\NormalTok{x }\OtherTok{\textless{}{-}}\NormalTok{ mtcars[}\DecValTok{1}\SpecialCharTok{:}\DecValTok{4}\NormalTok{, }\DecValTok{1}\SpecialCharTok{:}\DecValTok{5}\NormalTok{]}
\FunctionTok{tt}\NormalTok{(x)}
\end{Highlighting}
\end{Shaded}

\begin{table}[H]

\centering
\begin{tblr}[         %% tabularray outer open
]                     %% tabularray outer close
{                     %% tabularray inner open
colspec={Q[]Q[]Q[]Q[]Q[]},
}                     %% tabularray inner close
\toprule
mpg & cyl & disp & hp & drat \\ \midrule %% TinyTableHeader
21 & 6 & 160 & 110 & 3.9 \\
21 & 6 & 160 & 110 & 3.9 \\
22.8 & 4 & 108 & 93 & 3.85 \\
21.4 & 6 & 258 & 110 & 3.08 \\
\bottomrule
\end{tblr}
\end{table}

\subsection{Output formats}\label{output-formats}

\texttt{tinytable} can produce tables in HTML, Markdown, or LaTeX (PDF)
format. To choose, we use the \texttt{output} argument:

\begin{Shaded}
\begin{Highlighting}[]
\FunctionTok{tt}\NormalTok{(x, }\AttributeTok{output =} \StringTok{"html"}\NormalTok{)}
\FunctionTok{tt}\NormalTok{(x, }\AttributeTok{output =} \StringTok{"latex"}\NormalTok{)}
\FunctionTok{tt}\NormalTok{(x, }\AttributeTok{output =} \StringTok{"markdown"}\NormalTok{)}
\end{Highlighting}
\end{Shaded}

When calling \texttt{tinytable} from a Quarto or Rmarkdown document,
\texttt{tinytable} detects the output format automatically and generates
an HTML or LaTeX table as appropriate. This means that we do not need to
explicitly specify the \texttt{output} format.

\subsection{Themes}\label{themes}

\texttt{tinytable} offers a few basic themes out of the box:
``default'', ``striped'', ``grid'', ``void.'' Those themes can be
applied with the \texttt{theme} argument of the \texttt{tt()} function.
As we will see below, it is easy to go much beyond those basic settings
to customize your own tables. Here we only illustrate a few of the
simplest settings:

\begin{Shaded}
\begin{Highlighting}[]
\FunctionTok{tt}\NormalTok{(x, }\AttributeTok{theme =} \StringTok{"striped"}\NormalTok{)}
\end{Highlighting}
\end{Shaded}

\begin{table}[H]

\centering
\begin{tblr}[         %% tabularray outer open
]                     %% tabularray outer close
{                     %% tabularray inner open
colspec={Q[]Q[]Q[]Q[]Q[]},
row{even}={bg=black!5!white},
}                     %% tabularray inner close
\toprule
mpg & cyl & disp & hp & drat \\ \midrule %% TinyTableHeader
21 & 6 & 160 & 110 & 3.9 \\
21 & 6 & 160 & 110 & 3.9 \\
22.8 & 4 & 108 & 93 & 3.85 \\
21.4 & 6 & 258 & 110 & 3.08 \\
\bottomrule
\end{tblr}
\end{table}

\begin{Shaded}
\begin{Highlighting}[]
\FunctionTok{tt}\NormalTok{(x, }\AttributeTok{theme =} \StringTok{"grid"}\NormalTok{)}
\end{Highlighting}
\end{Shaded}

\begin{table}[H]

\centering
\begin{tblr}[         %% tabularray outer open
]                     %% tabularray outer close
{                     %% tabularray inner open
hlines,vlines,
colspec={Q[]Q[]Q[]Q[]Q[]},
hlines={},vlines={},
}                     %% tabularray inner close
mpg & cyl & disp & hp & drat \\
21 & 6 & 160 & 110 & 3.9 \\
21 & 6 & 160 & 110 & 3.9 \\
22.8 & 4 & 108 & 93 & 3.85 \\
21.4 & 6 & 258 & 110 & 3.08 \\
\end{tblr}
\end{table}

\begin{Shaded}
\begin{Highlighting}[]
\FunctionTok{tt}\NormalTok{(x, }\AttributeTok{theme =} \StringTok{"void"}\NormalTok{)}
\end{Highlighting}
\end{Shaded}

\begin{table}[H]

\centering
\begin{tblr}[         %% tabularray outer open
]                     %% tabularray outer close
{                     %% tabularray inner open
colspec={Q[]Q[]Q[]Q[]Q[]},
}                     %% tabularray inner close
mpg & cyl & disp & hp & drat \\
21 & 6 & 160 & 110 & 3.9 \\
21 & 6 & 160 & 110 & 3.9 \\
22.8 & 4 & 108 & 93 & 3.85 \\
21.4 & 6 & 258 & 110 & 3.08 \\
\end{tblr}
\end{table}

\subsection{Alignment}\label{alignment}

To align columns, we use a single string, where each letter represents a
column:

\begin{Shaded}
\begin{Highlighting}[]
\FunctionTok{tt}\NormalTok{(x, }\AttributeTok{align =} \StringTok{"ccrrl"}\NormalTok{)}
\end{Highlighting}
\end{Shaded}

\begin{table}[H]

\centering
\begin{tblr}[         %% tabularray outer open
]                     %% tabularray outer close
{                     %% tabularray inner open
colspec={Q[]Q[]Q[]Q[]Q[]},
column{1}={halign=c,},
column{2}={halign=c,},
column{3}={halign=r,},
column{4}={halign=r,},
column{5}={halign=l,},
}                     %% tabularray inner close
\toprule
mpg & cyl & disp & hp & drat \\ \midrule %% TinyTableHeader
21 & 6 & 160 & 110 & 3.9 \\
21 & 6 & 160 & 110 & 3.9 \\
22.8 & 4 & 108 & 93 & 3.85 \\
21.4 & 6 & 258 & 110 & 3.08 \\
\bottomrule
\end{tblr}
\end{table}

\subsection{Width}\label{width}

The \texttt{width} arguments accepts a number between 0 and 1,
indicating what proportion of the linewidth the table should cover:

\begin{Shaded}
\begin{Highlighting}[]
\FunctionTok{tt}\NormalTok{(x, }\AttributeTok{width =} \FloatTok{0.5}\NormalTok{)}
\end{Highlighting}
\end{Shaded}

\begin{table}[H]

\centering
\begin{tblr}[         %% tabularray outer open
]                     %% tabularray outer close
{                     %% tabularray inner open
width={0.5\linewidth},
colspec={X[]X[]X[]X[]X[]},
}                     %% tabularray inner close
\toprule
mpg & cyl & disp & hp & drat \\ \midrule %% TinyTableHeader
21 & 6 & 160 & 110 & 3.9 \\
21 & 6 & 160 & 110 & 3.9 \\
22.8 & 4 & 108 & 93 & 3.85 \\
21.4 & 6 & 258 & 110 & 3.08 \\
\bottomrule
\end{tblr}
\end{table}

\begin{Shaded}
\begin{Highlighting}[]
\FunctionTok{tt}\NormalTok{(x, }\AttributeTok{width =} \DecValTok{1}\NormalTok{)}
\end{Highlighting}
\end{Shaded}

\begin{table}[H]

\centering
\begin{tblr}[         %% tabularray outer open
]                     %% tabularray outer close
{                     %% tabularray inner open
width={1\linewidth},
colspec={X[]X[]X[]X[]X[]},
}                     %% tabularray inner close
\toprule
mpg & cyl & disp & hp & drat \\ \midrule %% TinyTableHeader
21 & 6 & 160 & 110 & 3.9 \\
21 & 6 & 160 & 110 & 3.9 \\
22.8 & 4 & 108 & 93 & 3.85 \\
21.4 & 6 & 258 & 110 & 3.08 \\
\bottomrule
\end{tblr}
\end{table}

\subsection{Line breaks and text
wrapping}\label{line-breaks-and-text-wrapping}

When the \texttt{width} argument is specified and a cell includes long
text, the text is automatically wrapped to match the table.

\begin{Shaded}
\begin{Highlighting}[]
\NormalTok{d }\OtherTok{\textless{}{-}} \FunctionTok{data.frame}\NormalTok{(}
  \AttributeTok{a =} \StringTok{"Sed ut perspiciatis unde omnis iste natus error sit voluptatem accusantium doloremque laudantium, totam rem aperiam, eaque ipsa quae ab illo inventore veritatis et quasi architecto beatae vitae"}\NormalTok{,}
  \AttributeTok{b =} \StringTok{"dicta sunt explicabo. Nemo enim ipsam voluptatem quia voluptas sit aspernatur aut odit aut fugit, sed quia consequuntur magni dolores eos"}
\NormalTok{)}
\FunctionTok{tt}\NormalTok{(d, }\AttributeTok{width =} \DecValTok{3}\SpecialCharTok{/}\DecValTok{4}\NormalTok{)}
\end{Highlighting}
\end{Shaded}

\begin{table}[H]

\centering
\begin{tblr}[         %% tabularray outer open
]                     %% tabularray outer close
{                     %% tabularray inner open
width={0.75\linewidth},
colspec={X[]X[]},
}                     %% tabularray inner close
\toprule
a & b \\ \midrule %% TinyTableHeader
Sed ut perspiciatis unde omnis iste natus error sit voluptatem accusantium doloremque laudantium, totam rem aperiam, eaque ipsa quae ab illo inventore veritatis et quasi architecto beatae vitae & dicta sunt explicabo. Nemo enim ipsam voluptatem quia voluptas sit aspernatur aut odit aut fugit, sed quia consequuntur magni dolores eos \\
\bottomrule
\end{tblr}
\end{table}

Manual line breaks work sligthly different in LaTeX (PDF) or HTML. This
table shows the two strategies. For HTML, we insert a
\texttt{\textless{}br\textgreater{}} tag. For LaTeX, we wrap the string
in curly braces \texttt{\{\}}, and then insert two (escaped)
backslashes:
\texttt{\textbackslash{}\textbackslash{}\textbackslash{}\textbackslash{}}

\begin{Shaded}
\begin{Highlighting}[]
\NormalTok{d }\OtherTok{\textless{}{-}} \FunctionTok{data.frame}\NormalTok{(}
  \StringTok{"\{Sed ut }\SpecialCharTok{\textbackslash{}\textbackslash{}\textbackslash{}\textbackslash{}}\StringTok{ perspiciatis unde\}"}\NormalTok{,}
  \StringTok{"dicta sunt\textless{}br\textgreater{} explicabo. Nemo"}
\NormalTok{) }\SpecialCharTok{|\textgreater{}} \FunctionTok{setNames}\NormalTok{(}\FunctionTok{c}\NormalTok{(}\StringTok{"LaTeX line break"}\NormalTok{, }\StringTok{"HTML line break"}\NormalTok{))}
\FunctionTok{tt}\NormalTok{(d, }\AttributeTok{width =} \DecValTok{1}\NormalTok{)}
\end{Highlighting}
\end{Shaded}

\begin{table}[H]

\centering
\begin{tblr}[         %% tabularray outer open
]                     %% tabularray outer close
{                     %% tabularray inner open
width={1\linewidth},
colspec={X[]X[]},
}                     %% tabularray inner close
\toprule
LaTeX line break & HTML line break \\ \midrule %% TinyTableHeader
{Sed ut \\ perspiciatis unde} & dicta sunt<br> explicabo. Nemo \\
\bottomrule
\end{tblr}
\end{table}

\subsection{Captions and
cross-references}\label{captions-and-cross-references}

In Quarto, one can specify captions and use cross-references using code
like this:

\begin{Shaded}
\begin{Highlighting}[]
\NormalTok{@tbl{-}blah shows that...}

\NormalTok{\textasciigrave{}\textasciigrave{}\textasciigrave{}\{r\}}
\NormalTok{\#| label: tbl{-}blah}
\NormalTok{\#| tbl{-}cap: "Blah blah blah"}
\NormalTok{library(tinytable)}
\NormalTok{tt(mtcars[1:4, 1:4])}
\NormalTok{\textasciigrave{}\textasciigrave{}\textasciigrave{}}
\end{Highlighting}
\end{Shaded}

And here is the rendered version of the code chunk above:

Table~\ref{tbl-blah} shows that\ldots{}

\begin{Shaded}
\begin{Highlighting}[]
\FunctionTok{library}\NormalTok{(tinytable)}
\FunctionTok{tt}\NormalTok{(mtcars[}\DecValTok{1}\SpecialCharTok{:}\DecValTok{4}\NormalTok{, }\DecValTok{1}\SpecialCharTok{:}\DecValTok{4}\NormalTok{], }\AttributeTok{placement =} \ConstantTok{NULL}\NormalTok{)}
\end{Highlighting}
\end{Shaded}

\begin{table}

\caption{\label{tbl-blah}Blah blah blah}

\centering{

\centering
\begin{tblr}[         %% tabularray outer open
]                     %% tabularray outer close
{                     %% tabularray inner open
colspec={Q[]Q[]Q[]Q[]},
}                     %% tabularray inner close
\toprule
mpg & cyl & disp & hp \\ \midrule %% TinyTableHeader
21 & 6 & 160 & 110 \\
21 & 6 & 160 & 110 \\
22.8 & 4 & 108 & 93 \\
21.4 & 6 & 258 & 110 \\
\bottomrule
\end{tblr}

}

\end{table}%

For standalone LaTeX tables, you can use the \texttt{caption} argument
like so:

\begin{Shaded}
\begin{Highlighting}[]
\FunctionTok{tt}\NormalTok{(x, }\AttributeTok{caption =} \StringTok{"Blah blah.}\SpecialCharTok{\textbackslash{}\textbackslash{}}\StringTok{label\{tbl{-}blah\}"}\NormalTok{)}
\end{Highlighting}
\end{Shaded}

Be aware that this more approach may not work well in Quarto or
Rmarkdown documents.

\subsection{Math}\label{math}

\begin{Shaded}
\begin{Highlighting}[]
\NormalTok{pkgload}\SpecialCharTok{::}\FunctionTok{load\_all}\NormalTok{()}
\end{Highlighting}
\end{Shaded}

\begin{verbatim}
i Loading tinytable
\end{verbatim}

\begin{Shaded}
\begin{Highlighting}[]
\NormalTok{dat }\OtherTok{\textless{}{-}} \FunctionTok{data.frame}\NormalTok{(}\AttributeTok{a =} \FunctionTok{c}\NormalTok{(}\StringTok{"x\^{}2 + y\^{}2 = z\^{}2$"}\NormalTok{, }\StringTok{"$}\SpecialCharTok{\textbackslash{}\textbackslash{}}\StringTok{frac\{1\}\{2\}$"}\NormalTok{))}
\FunctionTok{tt}\NormalTok{(dat, }\AttributeTok{output =} \StringTok{"html"}\NormalTok{) }\SpecialCharTok{|\textgreater{}} \FunctionTok{save\_tt}\NormalTok{(}\StringTok{"\textasciitilde{}/Downloads/junk.html"}\NormalTok{, }\AttributeTok{overwrite =} \ConstantTok{TRUE}\NormalTok{)}
\end{Highlighting}
\end{Shaded}

\section{Style}\label{style}

The main styling function for the \texttt{tinytable} package is
\texttt{style\_tt()}. Via this function, you can access three main
interfaces to customize tables:

\begin{enumerate}
\def\labelenumi{\arabic{enumi}.}
\tightlist
\item
  A general interface to frequently used style choices which works for
  both HTML and LaTeX (PDF): colors, font style and size, row and column
  spans, etc. This is accessed through several distinct arguments in the
  \texttt{style\_tt()} function, such as \texttt{italic},
  \texttt{color}, etc.
\item
  A specialized interface which allows users to use the
  \href{https://ctan.org/pkg/tabularray?lang=en}{powerful
  \texttt{tabularray} package} to customize LaTeX tables. This is
  accessed by passing \texttt{tabularray} settings as strings to the
  \texttt{tabularray\_inner} and \texttt{tabularray\_outer} arguments of
  \texttt{style\_tt()}.
\item
  A specialized interface which allows users to use the
  \href{https://getbootstrap.com/docs/5.3/content/tables/}{powerful
  \texttt{Bootstrap} framework} to customize HTML tables. This is
  accessed by passing CSS declarations and rules to the
  \texttt{bootstrap\_css} and \texttt{bootstrap\_css\_rule} arguments of
  \texttt{style\_tt()}.
\end{enumerate}

\subsection{Colors, lines, space, font, spans,
etc.}\label{colors-lines-space-font-spans-etc.}

These functions can be used to customize rows, columns, or individual
cells. They control many features, including:

\begin{itemize}
\tightlist
\item
  Text color
\item
  Background color
\item
  Widths
\item
  Heights
\item
  Alignment
\item
  Text Wrapping
\item
  Column and Row Spacing
\item
  Cell Merging
\item
  Multi-row or column spans
\item
  Border Styling
\item
  Font Styling
\item
  Header Customization
\end{itemize}

The \texttt{style\_*()} functions can modify individual cells, or entire
columns and rows. The portion of the table that is styled is determined
by the \texttt{i} (rows) and \texttt{j} (columns) arguments.

\subsection{Cells, rows, columns}\label{cells-rows-columns}

To style individual cells, we use the \texttt{style\_cell()} function.
The first two arguments---\texttt{i} and \texttt{j}---identify the cells
of interest, by row and column numbers respectively. To style a cell in
the 2nd row and 3rd column, we can do:

\begin{Shaded}
\begin{Highlighting}[]
\FunctionTok{tt}\NormalTok{(x) }\SpecialCharTok{|\textgreater{}}
  \FunctionTok{style\_tt}\NormalTok{(}
    \AttributeTok{i =} \DecValTok{2}\NormalTok{,}
    \AttributeTok{j =} \DecValTok{3}\NormalTok{,}
    \AttributeTok{background =} \StringTok{"black"}\NormalTok{,}
    \AttributeTok{color =} \StringTok{"white"}\NormalTok{)}
\end{Highlighting}
\end{Shaded}

\begin{table}[H]

\centering
\begin{tblr}[         %% tabularray outer open
]                     %% tabularray outer close
{                     %% tabularray inner open
colspec={Q[]Q[]Q[]Q[]Q[]},
cell{3}{3}={}{fg=white,bg=black,},
}                     %% tabularray inner close
\toprule
mpg & cyl & disp & hp & drat \\ \midrule %% TinyTableHeader
21 & 6 & 160 & 110 & 3.9 \\
21 & 6 & 160 & 110 & 3.9 \\
22.8 & 4 & 108 & 93 & 3.85 \\
21.4 & 6 & 258 & 110 & 3.08 \\
\bottomrule
\end{tblr}
\end{table}

The \texttt{i} and \texttt{j} accept vectors of integers to modify
several cells at once:

\begin{Shaded}
\begin{Highlighting}[]
\FunctionTok{tt}\NormalTok{(x) }\SpecialCharTok{|\textgreater{}}
  \FunctionTok{style\_tt}\NormalTok{(}
    \AttributeTok{i =} \DecValTok{2}\SpecialCharTok{:}\DecValTok{3}\NormalTok{,}
    \AttributeTok{j =} \FunctionTok{c}\NormalTok{(}\DecValTok{1}\NormalTok{, }\DecValTok{3}\NormalTok{, }\DecValTok{4}\NormalTok{),}
    \AttributeTok{italic =} \ConstantTok{TRUE}\NormalTok{,}
    \AttributeTok{color =} \StringTok{"red"}\NormalTok{)}
\end{Highlighting}
\end{Shaded}

\begin{table}[H]

\centering
\begin{tblr}[         %% tabularray outer open
]                     %% tabularray outer close
{                     %% tabularray inner open
colspec={Q[]Q[]Q[]Q[]Q[]},
cell{3,4}{1,3,4}={}{cmd=\textit,fg=red,},
}                     %% tabularray inner close
\toprule
mpg & cyl & disp & hp & drat \\ \midrule %% TinyTableHeader
21 & 6 & 160 & 110 & 3.9 \\
21 & 6 & 160 & 110 & 3.9 \\
22.8 & 4 & 108 & 93 & 3.85 \\
21.4 & 6 & 258 & 110 & 3.08 \\
\bottomrule
\end{tblr}
\end{table}

We can style all cells in a table by omitting both the \texttt{i} and
\texttt{j} arguments:

\begin{Shaded}
\begin{Highlighting}[]
\FunctionTok{tt}\NormalTok{(x) }\SpecialCharTok{|\textgreater{}} \FunctionTok{style\_tt}\NormalTok{(}\AttributeTok{color =} \StringTok{"blue"}\NormalTok{)}
\end{Highlighting}
\end{Shaded}

\begin{table}[H]

\centering
\begin{tblr}[         %% tabularray outer open
]                     %% tabularray outer close
{                     %% tabularray inner open
colspec={Q[]Q[]Q[]Q[]Q[]},
column{1,2,3,4,5}={fg=blue,},
}                     %% tabularray inner close
\toprule
mpg & cyl & disp & hp & drat \\ \midrule %% TinyTableHeader
21 & 6 & 160 & 110 & 3.9 \\
21 & 6 & 160 & 110 & 3.9 \\
22.8 & 4 & 108 & 93 & 3.85 \\
21.4 & 6 & 258 & 110 & 3.08 \\
\bottomrule
\end{tblr}
\end{table}

We can style entire rows by omitting the \texttt{j} argument:

\begin{Shaded}
\begin{Highlighting}[]
\FunctionTok{tt}\NormalTok{(x) }\SpecialCharTok{|\textgreater{}} \FunctionTok{style\_tt}\NormalTok{(}\AttributeTok{i =} \DecValTok{1}\SpecialCharTok{:}\DecValTok{2}\NormalTok{, }\AttributeTok{color =} \StringTok{"blue"}\NormalTok{)}
\end{Highlighting}
\end{Shaded}

\begin{table}[H]

\centering
\begin{tblr}[         %% tabularray outer open
]                     %% tabularray outer close
{                     %% tabularray inner open
colspec={Q[]Q[]Q[]Q[]Q[]},
row{2,3}={fg=blue,},
}                     %% tabularray inner close
\toprule
mpg & cyl & disp & hp & drat \\ \midrule %% TinyTableHeader
21 & 6 & 160 & 110 & 3.9 \\
21 & 6 & 160 & 110 & 3.9 \\
22.8 & 4 & 108 & 93 & 3.85 \\
21.4 & 6 & 258 & 110 & 3.08 \\
\bottomrule
\end{tblr}
\end{table}

We can style entire columns by omitting the \texttt{i} argument:

\begin{Shaded}
\begin{Highlighting}[]
\FunctionTok{tt}\NormalTok{(x) }\SpecialCharTok{|\textgreater{}} \FunctionTok{style\_tt}\NormalTok{(}\AttributeTok{j =} \FunctionTok{c}\NormalTok{(}\DecValTok{2}\NormalTok{, }\DecValTok{4}\NormalTok{), }\AttributeTok{bold =} \ConstantTok{TRUE}\NormalTok{)}
\end{Highlighting}
\end{Shaded}

\begin{table}[H]

\centering
\begin{tblr}[         %% tabularray outer open
]                     %% tabularray outer close
{                     %% tabularray inner open
colspec={Q[]Q[]Q[]Q[]Q[]},
column{2,4}={cmd=\bfseries,},
}                     %% tabularray inner close
\toprule
mpg & cyl & disp & hp & drat \\ \midrule %% TinyTableHeader
21 & 6 & 160 & 110 & 3.9 \\
21 & 6 & 160 & 110 & 3.9 \\
22.8 & 4 & 108 & 93 & 3.85 \\
21.4 & 6 & 258 & 110 & 3.08 \\
\bottomrule
\end{tblr}
\end{table}

Of course, we can also call the \texttt{style\_tt()} function several
times to apply different styles to different parts of the table:

\begin{Shaded}
\begin{Highlighting}[]
\FunctionTok{tt}\NormalTok{(x) }\SpecialCharTok{|\textgreater{}} 
  \FunctionTok{style\_tt}\NormalTok{(}\AttributeTok{i =} \DecValTok{1}\NormalTok{, }\AttributeTok{j =} \DecValTok{1}\SpecialCharTok{:}\DecValTok{2}\NormalTok{, }\AttributeTok{color =} \StringTok{"orange"}\NormalTok{) }\SpecialCharTok{|\textgreater{}}
  \FunctionTok{style\_tt}\NormalTok{(}\AttributeTok{i =} \DecValTok{1}\NormalTok{, }\AttributeTok{j =} \DecValTok{3}\SpecialCharTok{:}\DecValTok{4}\NormalTok{, }\AttributeTok{color =} \StringTok{"green"}\NormalTok{)}
\end{Highlighting}
\end{Shaded}

\begin{table}[H]

\centering
\begin{tblr}[         %% tabularray outer open
]                     %% tabularray outer close
{                     %% tabularray inner open
colspec={Q[]Q[]Q[]Q[]Q[]},
cell{2}{1,2}={}{fg=orange,},
cell{2}{3,4}={}{fg=green,},
}                     %% tabularray inner close
\toprule
mpg & cyl & disp & hp & drat \\ \midrule %% TinyTableHeader
21 & 6 & 160 & 110 & 3.9 \\
21 & 6 & 160 & 110 & 3.9 \\
22.8 & 4 & 108 & 93 & 3.85 \\
21.4 & 6 & 258 & 110 & 3.08 \\
\bottomrule
\end{tblr}
\end{table}

\subsection{Colors}\label{colors}

The \texttt{color} and \texttt{background} arguments in the
\texttt{style\_tt()} function are used for specifying the text color and
the background color for cells of a table created by the \texttt{tt()}
function. This argument plays a crucial role in enhancing the visual
appeal and readability of the table, whether it's rendered in LaTeX or
HTML format. The way we specify colors differs slightly between the two
formats:

For HTML Output:

\begin{itemize}
\tightlist
\item
  Hex Codes: You can specify colors using hexadecimal codes, which
  consist of a \texttt{\#} followed by 6 characters (e.g.,
  \texttt{\#CC79A7}). This allows for a wide range of colors.
\item
  Keywords: There's also the option to use color keywords for
  convenience. The supported keywords are basic color names like
  \texttt{black}, \texttt{red}, \texttt{blue}, etc.
\end{itemize}

For LaTeX Output:

\begin{itemize}
\tightlist
\item
  Hexadecimal Codes: Similar to HTML, you can use hexadecimal codes.
  However, in LaTeX, you need to include these codes as strings (e.g.,
  \texttt{"\#CC79A7"}).
\item
  Keywords: LaTeX supports a different set of color keywords, which
  include standard colors like \texttt{black}, \texttt{red},
  \texttt{blue}, as well as additional ones like \texttt{cyan},
  \texttt{darkgray}, \texttt{lightgray}, etc.
\item
  Color Blending: An advanced feature in LaTeX is color blending, which
  can be achieved using the \texttt{xcolor} package. You can blend
  colors by specifying ratios (e.g., \texttt{white!80!blue} or
  \texttt{green!20!red}).
\item
  Luminance Levels: \href{https://ctan.org/pkg/ninecolors?lang=en}{The
  \texttt{ninecolors} package in LaTeX} offers colors with predefined
  luminance levels, allowing for more nuanced color choices (e.g.,
  ``azure4'', ``magenta8'').
\end{itemize}

Note that the keywords used in LaTeX and HTML are slightly different.

\begin{Shaded}
\begin{Highlighting}[]
\FunctionTok{tt}\NormalTok{(x) }\SpecialCharTok{|\textgreater{}} \FunctionTok{style\_tt}\NormalTok{(}\AttributeTok{i =} \DecValTok{1}\SpecialCharTok{:}\DecValTok{4}\NormalTok{, }\AttributeTok{j =} \DecValTok{1}\NormalTok{, }\AttributeTok{color =} \StringTok{"\#FF5733"}\NormalTok{)}
\end{Highlighting}
\end{Shaded}

\begin{table}[H]

\centering
\begin{tblr}[         %% tabularray outer open
]                     %% tabularray outer close
{                     %% tabularray inner open
colspec={Q[]Q[]Q[]Q[]Q[]},
cell{2,3,4,5}{1}={}{fg=cFF5733,},
}                     %% tabularray inner close
\tinytableDefineColor{cFF5733}{HTML}{FF5733}
\toprule
mpg & cyl & disp & hp & drat \\ \midrule %% TinyTableHeader
21 & 6 & 160 & 110 & 3.9 \\
21 & 6 & 160 & 110 & 3.9 \\
22.8 & 4 & 108 & 93 & 3.85 \\
21.4 & 6 & 258 & 110 & 3.08 \\
\bottomrule
\end{tblr}
\end{table}

Note that when using Hex codes in a LaTeX table, we need extra
declarations in the LaTeX preamble. See \texttt{?tt} for details.

\subsection{Spanning cells}\label{spanning-cells}

Sometimes, it can be useful to make a cell stretch across multiple
colums, for example when we want to insert a label. To achieve this, we
can use the \texttt{colspan} argument. Here, we make the 2nd cell of the
2nd row stretch across three columns:

\begin{Shaded}
\begin{Highlighting}[]
\FunctionTok{tt}\NormalTok{(x)}\SpecialCharTok{|\textgreater{}} \FunctionTok{style\_tt}\NormalTok{(}
  \AttributeTok{i =} \DecValTok{2}\NormalTok{, }\AttributeTok{j =} \DecValTok{2}\NormalTok{,}
  \AttributeTok{colspan =} \DecValTok{3}\NormalTok{,}
  \AttributeTok{align =} \StringTok{"c"}\NormalTok{,}
  \AttributeTok{color =} \StringTok{"white"}\NormalTok{,}
  \AttributeTok{background =} \StringTok{"black"}\NormalTok{)}
\end{Highlighting}
\end{Shaded}

\begin{table}[H]

\centering
\begin{tblr}[         %% tabularray outer open
]                     %% tabularray outer close
{                     %% tabularray inner open
colspec={Q[]Q[]Q[]Q[]Q[]},
cell{3}{2}={c=3}{halign=c,fg=white,bg=black,},
}                     %% tabularray inner close
\toprule
mpg & cyl & disp & hp & drat \\ \midrule %% TinyTableHeader
21 & 6 & 160 & 110 & 3.9 \\
21 & 6 & 160 & 110 & 3.9 \\
22.8 & 4 & 108 & 93 & 3.85 \\
21.4 & 6 & 258 & 110 & 3.08 \\
\bottomrule
\end{tblr}
\end{table}

Here is the original table for comparison:

\begin{Shaded}
\begin{Highlighting}[]
\FunctionTok{tt}\NormalTok{(x)}
\end{Highlighting}
\end{Shaded}

\begin{table}[H]

\centering
\begin{tblr}[         %% tabularray outer open
]                     %% tabularray outer close
{                     %% tabularray inner open
colspec={Q[]Q[]Q[]Q[]Q[]},
}                     %% tabularray inner close
\toprule
mpg & cyl & disp & hp & drat \\ \midrule %% TinyTableHeader
21 & 6 & 160 & 110 & 3.9 \\
21 & 6 & 160 & 110 & 3.9 \\
22.8 & 4 & 108 & 93 & 3.85 \\
21.4 & 6 & 258 & 110 & 3.08 \\
\bottomrule
\end{tblr}
\end{table}

\subsection{Headers}\label{headers}

The header can be omitted from the table by deleting the column names in
the \texttt{x} data frame:

\begin{Shaded}
\begin{Highlighting}[]
\NormalTok{k }\OtherTok{\textless{}{-}}\NormalTok{ x}
\FunctionTok{colnames}\NormalTok{(k) }\OtherTok{\textless{}{-}} \ConstantTok{NULL}
\FunctionTok{tt}\NormalTok{(k)}
\end{Highlighting}
\end{Shaded}

\begin{table}[H]

\centering
\begin{tblr}[         %% tabularray outer open
]                     %% tabularray outer close
{                     %% tabularray inner open
colspec={Q[]Q[]Q[]Q[]Q[]},
}                     %% tabularray inner close
\toprule
21 & 6 & 160 & 110 & 3.9 \\
21 & 6 & 160 & 110 & 3.9 \\
22.8 & 4 & 108 & 93 & 3.85 \\
21.4 & 6 & 258 & 110 & 3.08 \\
\bottomrule
\end{tblr}
\end{table}

\section{Groups and labels}\label{groups-and-labels}

The \texttt{group\_tt()} function can label groups of rows (\texttt{i})
or columns (\texttt{j}).

\subsection{Rows}\label{rows}

The \texttt{i} argument accepts a named list of integers. The numbers
identify the positions where row group labels are to be inserted. The
names includes the text that should be inserted:

\begin{Shaded}
\begin{Highlighting}[]
\NormalTok{dat }\OtherTok{\textless{}{-}}\NormalTok{ mtcars[}\DecValTok{1}\SpecialCharTok{:}\DecValTok{9}\NormalTok{, }\DecValTok{1}\SpecialCharTok{:}\DecValTok{8}\NormalTok{]}

\FunctionTok{tt}\NormalTok{(dat) }\SpecialCharTok{|\textgreater{}} 
  \FunctionTok{group\_tt}\NormalTok{(}\AttributeTok{i =} \FunctionTok{list}\NormalTok{(}
    \StringTok{"I like (fake) hamburgers"} \OtherTok{=} \DecValTok{3}\NormalTok{,}
    \StringTok{"She prefers halloumi"} \OtherTok{=} \DecValTok{4}\NormalTok{,}
    \StringTok{"They love tofu"} \OtherTok{=} \DecValTok{7}\NormalTok{))}
\end{Highlighting}
\end{Shaded}

\begin{table}[H]

\centering
\begin{tblr}[         %% tabularray outer open
]                     %% tabularray outer close
{                     %% tabularray inner open
colspec={Q[]Q[]Q[]Q[]Q[]Q[]Q[]Q[]},
cell{4}{1}={c=8}{},cell{6}{1}={c=8}{},cell{10}{1}={c=8}{},
cell{1}{1}={preto={\hspace{1em}}},cell{2}{1}={preto={\hspace{1em}}},cell{3}{1}={preto={\hspace{1em}}},cell{5}{1}={preto={\hspace{1em}}},cell{7}{1}={preto={\hspace{1em}}},cell{8}{1}={preto={\hspace{1em}}},cell{9}{1}={preto={\hspace{1em}}},cell{11}{1}={preto={\hspace{1em}}},cell{12}{1}={preto={\hspace{1em}}},cell{13}{1}={preto={\hspace{1em}}},
}                     %% tabularray inner close
\toprule
mpg & cyl & disp & hp & drat & wt & qsec & vs \\ \midrule %% TinyTableHeader
21 & 6 & 160 & 110 & 3.9 & 2.62 & 16.46 & 0 \\
21 & 6 & 160 & 110 & 3.9 & 2.875 & 17.02 & 0 \\
I like (fake) hamburgers &&&&&&&& \\
22.8 & 4 & 108 & 93 & 3.85 & 2.32 & 18.61 & 1 \\
She prefers halloumi &&&&&&&& \\
21.4 & 6 & 258 & 110 & 3.08 & 3.215 & 19.44 & 1 \\
18.7 & 8 & 360 & 175 & 3.15 & 3.44 & 17.02 & 0 \\
18.1 & 6 & 225 & 105 & 2.76 & 3.46 & 20.22 & 1 \\
They love tofu &&&&&&&& \\
14.3 & 8 & 360 & 245 & 3.21 & 3.57 & 15.84 & 0 \\
24.4 & 4 & 146.7 & 62 & 3.69 & 3.19 & 20 & 1 \\
22.8 & 4 & 140.8 & 95 & 3.92 & 3.15 & 22.9 & 1 \\
\bottomrule
\end{tblr}
\end{table}

The \texttt{group\_tt()} function only includes a few arguments:
\texttt{x}, \texttt{i}, \texttt{j}, and \texttt{indent}. But whenever we
call \texttt{group\_tt()}, the function will automatically apply a
\texttt{style\_tt()} call to all the new group labels, using any extra
argument supplied to \texttt{group\_tt()} (arguments are pushed via
\texttt{...}). This means that we can apply all the usual stying options
to row labels:

\begin{Shaded}
\begin{Highlighting}[]
\FunctionTok{tt}\NormalTok{(dat) }\SpecialCharTok{|\textgreater{}} 
  \FunctionTok{group\_tt}\NormalTok{(}
    \AttributeTok{align =} \StringTok{"c"}\NormalTok{,}
    \AttributeTok{color =} \StringTok{"white"}\NormalTok{,}
    \AttributeTok{background =} \StringTok{"gray"}\NormalTok{,}
    \AttributeTok{bold =} \ConstantTok{TRUE}\NormalTok{,}
    \AttributeTok{i =} \FunctionTok{list}\NormalTok{(}
      \StringTok{"I like (fake) hamburgers"} \OtherTok{=} \DecValTok{3}\NormalTok{,}
      \StringTok{"She prefers halloumi"} \OtherTok{=} \DecValTok{4}\NormalTok{,}
      \StringTok{"They love tofu"} \OtherTok{=} \DecValTok{7}\NormalTok{))}
\end{Highlighting}
\end{Shaded}

\begin{table}[H]

\centering
\begin{tblr}[         %% tabularray outer open
]                     %% tabularray outer close
{                     %% tabularray inner open
colspec={Q[]Q[]Q[]Q[]Q[]Q[]Q[]Q[]},
cell{4}{1}={c=8}{},cell{6}{1}={c=8}{},cell{10}{1}={c=8}{},
cell{1}{1}={preto={\hspace{1em}}},cell{2}{1}={preto={\hspace{1em}}},cell{3}{1}={preto={\hspace{1em}}},cell{5}{1}={preto={\hspace{1em}}},cell{7}{1}={preto={\hspace{1em}}},cell{8}{1}={preto={\hspace{1em}}},cell{9}{1}={preto={\hspace{1em}}},cell{11}{1}={preto={\hspace{1em}}},cell{12}{1}={preto={\hspace{1em}}},cell{13}{1}={preto={\hspace{1em}}},
row{4,6,10}={cmd=\bfseries,halign=c,fg=white,bg=gray,},
}                     %% tabularray inner close
\toprule
mpg & cyl & disp & hp & drat & wt & qsec & vs \\ \midrule %% TinyTableHeader
21 & 6 & 160 & 110 & 3.9 & 2.62 & 16.46 & 0 \\
21 & 6 & 160 & 110 & 3.9 & 2.875 & 17.02 & 0 \\
I like (fake) hamburgers &&&&&&&& \\
22.8 & 4 & 108 & 93 & 3.85 & 2.32 & 18.61 & 1 \\
She prefers halloumi &&&&&&&& \\
21.4 & 6 & 258 & 110 & 3.08 & 3.215 & 19.44 & 1 \\
18.7 & 8 & 360 & 175 & 3.15 & 3.44 & 17.02 & 0 \\
18.1 & 6 & 225 & 105 & 2.76 & 3.46 & 20.22 & 1 \\
They love tofu &&&&&&&& \\
14.3 & 8 & 360 & 245 & 3.21 & 3.57 & 15.84 & 0 \\
24.4 & 4 & 146.7 & 62 & 3.69 & 3.19 & 20 & 1 \\
22.8 & 4 & 140.8 & 95 & 3.92 & 3.15 & 22.9 & 1 \\
\bottomrule
\end{tblr}
\end{table}

\subsection{Columns}\label{columns}

The syntax for column groups is very similar, but we use the \texttt{j}
argument instead. The named list specifies the labels to appear in
column-spanning labels, and the values must be a vector of consecutive
and non-overlapping integers that indicate which columns are associated
to which labels:

\begin{Shaded}
\begin{Highlighting}[]
\FunctionTok{tt}\NormalTok{(dat) }\SpecialCharTok{|\textgreater{}} 
  \FunctionTok{group\_tt}\NormalTok{(}
    \AttributeTok{j =} \FunctionTok{list}\NormalTok{(}
      \StringTok{"Hamburgers"} \OtherTok{=} \DecValTok{1}\SpecialCharTok{:}\DecValTok{3}\NormalTok{,}
      \StringTok{"Halloumi"} \OtherTok{=} \DecValTok{4}\SpecialCharTok{:}\DecValTok{5}\NormalTok{,}
      \StringTok{"Tofu"} \OtherTok{=} \DecValTok{7}\NormalTok{))}
\end{Highlighting}
\end{Shaded}

\begin{table}[H]

\centering
\begin{tblr}[         %% tabularray outer open
]                     %% tabularray outer close
{                     %% tabularray inner open
colspec={Q[]Q[]Q[]Q[]Q[]Q[]Q[]Q[]},
cell{1}{1}={c=3}{halign=c,},
cell{1}{4}={c=2}{halign=c,},
cell{1}{7}={c=1}{halign=c,},
}                     %% tabularray inner close
\toprule
Hamburgers &  &  & Halloumi &  &  & Tofu &  \\ \cmidrule[lr]{1-3}\cmidrule[lr]{4-5}\cmidrule[lr]{7-7}
mpg & cyl & disp & hp & drat & wt & qsec & vs \\ \midrule %% TinyTableHeader
21 & 6 & 160 & 110 & 3.9 & 2.62 & 16.46 & 0 \\
21 & 6 & 160 & 110 & 3.9 & 2.875 & 17.02 & 0 \\
22.8 & 4 & 108 & 93 & 3.85 & 2.32 & 18.61 & 1 \\
21.4 & 6 & 258 & 110 & 3.08 & 3.215 & 19.44 & 1 \\
18.7 & 8 & 360 & 175 & 3.15 & 3.44 & 17.02 & 0 \\
18.1 & 6 & 225 & 105 & 2.76 & 3.46 & 20.22 & 1 \\
14.3 & 8 & 360 & 245 & 3.21 & 3.57 & 15.84 & 0 \\
24.4 & 4 & 146.7 & 62 & 3.69 & 3.19 & 20 & 1 \\
22.8 & 4 & 140.8 & 95 & 3.92 & 3.15 & 22.9 & 1 \\
\bottomrule
\end{tblr}
\end{table}

As above, we can pass additional styling options to the
\texttt{style\_tt()} function automatically via \texttt{...}. This means
that all the arguments like \texttt{italic}, \texttt{bold},
\texttt{color} and friends can be used to style spanning column headers:

\begin{Shaded}
\begin{Highlighting}[]
\FunctionTok{tt}\NormalTok{(dat) }\SpecialCharTok{|\textgreater{}} 
  \FunctionTok{group\_tt}\NormalTok{(}\AttributeTok{color =} \StringTok{"teal"}\NormalTok{, }\AttributeTok{italic =} \ConstantTok{TRUE}\NormalTok{,}
    \AttributeTok{j =} \FunctionTok{list}\NormalTok{(}\StringTok{"Hamburgers"} \OtherTok{=} \DecValTok{1}\SpecialCharTok{:}\DecValTok{3}\NormalTok{,}
             \StringTok{"Halloumi"} \OtherTok{=} \DecValTok{4}\SpecialCharTok{:}\DecValTok{5}\NormalTok{,}
             \StringTok{"Tofu"} \OtherTok{=} \DecValTok{7}\NormalTok{)) }\SpecialCharTok{|\textgreater{}}
  \FunctionTok{group\_tt}\NormalTok{(}\AttributeTok{align =} \StringTok{"c"}\NormalTok{, }\AttributeTok{color =} \StringTok{"white"}\NormalTok{, }\AttributeTok{background =} \StringTok{"teal"}\NormalTok{, }\AttributeTok{bold =} \ConstantTok{TRUE}\NormalTok{,}
    \AttributeTok{i =} \FunctionTok{list}\NormalTok{(}\StringTok{"I like (fake) hamburgers"} \OtherTok{=} \DecValTok{3}\NormalTok{,}
             \StringTok{"She prefers halloumi"} \OtherTok{=} \DecValTok{4}\NormalTok{,}
             \StringTok{"They love tofu"} \OtherTok{=} \DecValTok{7}\NormalTok{))}
\end{Highlighting}
\end{Shaded}

\begin{table}[H]

\centering
\begin{tblr}[         %% tabularray outer open
]                     %% tabularray outer close
{                     %% tabularray inner open
colspec={Q[]Q[]Q[]Q[]Q[]Q[]Q[]Q[]},
cell{1}{1}={c=3}{cmd=\textit,halign=c,fg=teal,},
cell{1}{4}={c=2}{cmd=\textit,halign=c,fg=teal,},
cell{1}{7}={c=1}{cmd=\textit,halign=c,fg=teal,},
cell{5}{1}={c=8}{},cell{7}{1}={c=8}{},cell{11}{1}={c=8}{},
cell{1}{1}={preto={\hspace{1em}}},cell{2}{1}={preto={\hspace{1em}}},cell{3}{1}={preto={\hspace{1em}}},cell{4}{1}={preto={\hspace{1em}}},cell{6}{1}={preto={\hspace{1em}}},cell{8}{1}={preto={\hspace{1em}}},cell{9}{1}={preto={\hspace{1em}}},cell{10}{1}={preto={\hspace{1em}}},cell{12}{1}={preto={\hspace{1em}}},cell{13}{1}={preto={\hspace{1em}}},cell{14}{1}={preto={\hspace{1em}}},
row{5,7,11}={cmd=\bfseries,halign=c,fg=white,bg=teal,},
}                     %% tabularray inner close
\toprule
Hamburgers &  &  & Halloumi &  &  & Tofu &  \\ \cmidrule[lr]{1-3}\cmidrule[lr]{4-5}\cmidrule[lr]{7-7}
mpg & cyl & disp & hp & drat & wt & qsec & vs \\ \midrule %% TinyTableHeader
21 & 6 & 160 & 110 & 3.9 & 2.62 & 16.46 & 0 \\
21 & 6 & 160 & 110 & 3.9 & 2.875 & 17.02 & 0 \\
I like (fake) hamburgers &&&&&&&& \\
22.8 & 4 & 108 & 93 & 3.85 & 2.32 & 18.61 & 1 \\
She prefers halloumi &&&&&&&& \\
21.4 & 6 & 258 & 110 & 3.08 & 3.215 & 19.44 & 1 \\
18.7 & 8 & 360 & 175 & 3.15 & 3.44 & 17.02 & 0 \\
18.1 & 6 & 225 & 105 & 2.76 & 3.46 & 20.22 & 1 \\
They love tofu &&&&&&&& \\
14.3 & 8 & 360 & 245 & 3.21 & 3.57 & 15.84 & 0 \\
24.4 & 4 & 146.7 & 62 & 3.69 & 3.19 & 20 & 1 \\
22.8 & 4 & 140.8 & 95 & 3.92 & 3.15 & 22.9 & 1 \\
\bottomrule
\end{tblr}
\end{table}

\section{HTML customization}\label{html-customization}

The HTML customization options described in this section are not
available for LaTeX (or PDF) documents. Please refer to the web
documentation to read this part of the tutorial.

\subsection{Themes}\label{themes-1}

\subsection{CSS declarations}\label{css-declarations}

\subsection{CSS rules}\label{css-rules}

\section{LaTeX / PDF customization}\label{latex-pdf-customization}

\subsection{Preamble}\label{preamble}

In Rmarkdown and Quarto documents, \texttt{tinytable} will automatically
populate your LaTeX preamble with the necessary packages and commands.
When creating your own LaTeX documents, you should insert these commands
in the preamble:

\begin{Shaded}
\begin{Highlighting}[]
\BuiltInTok{\textbackslash{}usepackage}\NormalTok{\{}\ExtensionTok{float}\NormalTok{\}}
\BuiltInTok{\textbackslash{}usepackage}\NormalTok{\{}\ExtensionTok{codehigh}\NormalTok{\}}
\BuiltInTok{\textbackslash{}usepackage}\NormalTok{\{}\ExtensionTok{tabularray}\NormalTok{\}}
\FunctionTok{\textbackslash{}UseTblrLibrary}\NormalTok{\{booktabs\}}
\FunctionTok{\textbackslash{}NewTableCommand}\NormalTok{\{}\FunctionTok{\textbackslash{}tinytableDefineColor}\NormalTok{\}[3]\{}\FunctionTok{\textbackslash{}definecolor}\NormalTok{\{\#1\}\{\#2\}\{\#3\}\}}
\end{Highlighting}
\end{Shaded}

\subsection{\texorpdfstring{Introduction to
\texttt{tabularray}}{Introduction to tabularray}}\label{introduction-to-tabularray}

\texttt{tabularray} offers a robust solution for creating and managing
tables in LaTeX, standing out for its flexibility and ease of use. It
excels in handling complex table layouts and offers enhanced
functionality compared to traditional LaTeX table environments. This
package is particularly useful for users requiring advanced table
features, such as complex cell formatting, color management, and
versatile table structures.

A key feature of Tabularray is its separation of style from content.
This approach allows users to define the look and feel of their tables
(such as color, borders, and text alignment) independently from the
actual data within the table. This separation simplifies the process of
formatting tables and enhances the clarity and maintainability of LaTeX
code. The \texttt{tabularray} documentation is fantastic. It will teach
you how to customize virtually every aspect of your tables:
\url{https://ctan.org/pkg/tabularray?lang=en}

Tabularray introduces a streamlined interface for specifying table
settings. It employs two types of settings blocks: Inner and Outer. The
Outer block is used for settings that apply to the entire table, like
overall alignment, while the Inner block handles settings for specific
elements like columns, rows, and cells. The \texttt{style\_tt()}
function includes \texttt{tabularray\_inner} and
\texttt{tabularray\_outer} arguments to set these respective features.

Consider this \texttt{tabularray} example, which illustrates the use of
inner settings:

\begin{Shaded}
\begin{Highlighting}[]
\KeywordTok{\textbackslash{}begin}\NormalTok{\{}\ExtensionTok{table}\NormalTok{\}}
\FunctionTok{\textbackslash{}centering}
\KeywordTok{\textbackslash{}begin}\NormalTok{\{}\ExtensionTok{tblr}\NormalTok{\}[         }\CommentTok{\%\% tabularray outer open}
\NormalTok{]                     }\CommentTok{\%\% tabularray outer close}
\NormalTok{\{                     }\CommentTok{\%\% tabularray inner open}
\NormalTok{column\{1{-}4\}=\{halign=c\},}
\NormalTok{hlines = \{bg=white\},}
\NormalTok{vlines = \{bg=white\},}
\NormalTok{cell\{1,6\}\{odd\} = \{bg=teal7\},}
\NormalTok{cell\{1,6\}\{even\} = \{bg=green7\},}
\NormalTok{cell\{2,4\}\{1,4\} = \{bg=red7\},}
\NormalTok{cell\{3,5\}\{1,4\} = \{bg=purple7\},}
\NormalTok{cell\{2\}\{2\} = \{r=4,c=2\}\{bg=azure7\},}
\NormalTok{\}                     }\CommentTok{\%\% tabularray inner close}
\NormalTok{mpg \& cyl \& disp \& hp }\FunctionTok{\textbackslash{}\textbackslash{}}
\NormalTok{21 \& 6 \& 160 \& 110 }\FunctionTok{\textbackslash{}\textbackslash{}}
\NormalTok{21 \& 6 \& 160 \& 110 }\FunctionTok{\textbackslash{}\textbackslash{}}
\NormalTok{22.8 \& 4 \& 108 \& 93 }\FunctionTok{\textbackslash{}\textbackslash{}}
\NormalTok{21.4 \& 6 \& 258 \& 110 }\FunctionTok{\textbackslash{}\textbackslash{}}
\NormalTok{18.7 \& 8 \& 360 \& 175 }\FunctionTok{\textbackslash{}\textbackslash{}}
\KeywordTok{\textbackslash{}end}\NormalTok{\{}\ExtensionTok{tblr}\NormalTok{\}}
\KeywordTok{\textbackslash{}end}\NormalTok{\{}\ExtensionTok{table}\NormalTok{\}}
\end{Highlighting}
\end{Shaded}

The Inner block, enclosed in \texttt{\{\}}, defines specific styles like
column formats (\texttt{column\{1-4\}=\{halign=c\}}), horizontal and
vertical line colors (\texttt{hlines=\{fg=white\}},
\texttt{vlines=\{fg=white\}}), and cell colorations
(\texttt{cell\{1,6\}\{odd\}=\{bg=teal7\}}, etc.). The last line of the
inner block also species that the second cell of row 2
(\texttt{cell\{2\}\{2\}}) should span 4 rows and 2 columns
(\texttt{\{r=4,c=3\}}), be centered (\texttt{halign=c}), and with a
background color with the 7th luminance level of the azure color
(\texttt{bg=azure7}).

We can create this code easily by passing a string to the
\texttt{tabularray\_inner} argument of the \texttt{style\_tt()}
function:

\begin{Shaded}
\begin{Highlighting}[]
\NormalTok{inner }\OtherTok{\textless{}{-}} \StringTok{"}
\StringTok{column\{1{-}4\}=\{halign=c\},}
\StringTok{hlines = \{fg=white\},}
\StringTok{vlines = \{fg=white\},}
\StringTok{cell\{1,6\}\{odd\} = \{bg=teal7\},}
\StringTok{cell\{1,6\}\{even\} = \{bg=green7\},}
\StringTok{cell\{2,4\}\{1,4\} = \{bg=red7\},}
\StringTok{cell\{3,5\}\{1,4\} = \{bg=purple7\},}
\StringTok{cell\{2\}\{2\} = \{r=4,c=2\}\{bg=azure7\},}
\StringTok{"}
\NormalTok{mtcars[}\DecValTok{1}\SpecialCharTok{:}\DecValTok{5}\NormalTok{, }\DecValTok{1}\SpecialCharTok{:}\DecValTok{4}\NormalTok{] }\SpecialCharTok{|\textgreater{}}
  \FunctionTok{tt}\NormalTok{(}\AttributeTok{output =} \StringTok{"latex"}\NormalTok{, }\AttributeTok{theme =} \StringTok{"void"}\NormalTok{) }\SpecialCharTok{|\textgreater{}}
  \FunctionTok{style\_tt}\NormalTok{(}\AttributeTok{tabularray\_inner =}\NormalTok{ inner)}
\end{Highlighting}
\end{Shaded}

\begin{table}[H]

\centering
\begin{tblr}[         %% tabularray outer open
]                     %% tabularray outer close
{                     %% tabularray inner open
colspec={Q[]Q[]Q[]Q[]},
column{1,2,3,4}={},
column{1-4}={halign=c},
hlines = {fg=white},
vlines = {fg=white},
cell{1,6}{odd} = {bg=teal7},
cell{1,6}{even} = {bg=green7},
cell{2,4}{1,4} = {bg=red7},
cell{3,5}{1,4} = {bg=purple7},
cell{2}{2} = {r=4,c=2}{bg=azure7},
}                     %% tabularray inner close
mpg & cyl & disp & hp \\
21 & 6 & 160 & 110 \\
21 & 6 & 160 & 110 \\
22.8 & 4 & 108 & 93 \\
21.4 & 6 & 258 & 110 \\
18.7 & 8 & 360 & 175 \\
\end{tblr}
\end{table}

\subsection{\texorpdfstring{\texttt{tabularray}
keys}{tabularray keys}}\label{tabularray-keys}

Inner specifications:

\begin{longtable}[]{@{}
  >{\raggedright\arraybackslash}p{(\columnwidth - 4\tabcolsep) * \real{0.1294}}
  >{\raggedright\arraybackslash}p{(\columnwidth - 4\tabcolsep) * \real{0.6941}}
  >{\raggedright\arraybackslash}p{(\columnwidth - 4\tabcolsep) * \real{0.1765}}@{}}
\toprule\noalign{}
\begin{minipage}[b]{\linewidth}\raggedright
Key
\end{minipage} & \begin{minipage}[b]{\linewidth}\raggedright
Description and Values
\end{minipage} & \begin{minipage}[b]{\linewidth}\raggedright
Initial Value
\end{minipage} \\
\midrule\noalign{}
\endhead
\bottomrule\noalign{}
\endlastfoot
\texttt{rulesep} & space between two hlines or vlines & \texttt{2pt} \\
\texttt{stretch} & stretch ratio for struts added to cell text &
\texttt{1} \\
\texttt{abovesep} & set vertical space above every row & \texttt{2pt} \\
\texttt{belowsep} & set vertical space below every row & \texttt{2pt} \\
\texttt{rowsep} & set vertical space above and below every row &
\texttt{2pt} \\
\texttt{leftsep} & set horizontal space to the left of every column &
\texttt{6pt} \\
\texttt{rightsep} & set horizontal space to the right of every column &
\texttt{6pt} \\
\texttt{colsep} & set horizontal space to both sides of every column &
\texttt{6pt} \\
\texttt{hspan} & horizontal span algorithm: \texttt{default},
\texttt{even}, or \texttt{minimal} & \texttt{default} \\
\texttt{vspan} & vertical span algorithm: \texttt{default} or
\texttt{even} & \texttt{default} \\
\texttt{baseline} & set the baseline of the table & \texttt{m} \\
\end{longtable}

Outer specifications:

\begin{longtable}[]{@{}
  >{\raggedright\arraybackslash}p{(\columnwidth - 4\tabcolsep) * \real{0.1467}}
  >{\raggedright\arraybackslash}p{(\columnwidth - 4\tabcolsep) * \real{0.6533}}
  >{\raggedright\arraybackslash}p{(\columnwidth - 4\tabcolsep) * \real{0.2000}}@{}}
\toprule\noalign{}
\begin{minipage}[b]{\linewidth}\raggedright
Key
\end{minipage} & \begin{minipage}[b]{\linewidth}\raggedright
Description and Values
\end{minipage} & \begin{minipage}[b]{\linewidth}\raggedright
Initial Value
\end{minipage} \\
\midrule\noalign{}
\endhead
\bottomrule\noalign{}
\endlastfoot
\texttt{baseline} & set the baseline of the table & \texttt{m} \\
\texttt{long} & change the table to a long table & None \\
\texttt{tall} & change the table to a tall table & None \\
\texttt{expand} & you need this key to use verb commands & None \\
\end{longtable}

Cells:

\begin{longtable}[]{@{}
  >{\raggedright\arraybackslash}p{(\columnwidth - 4\tabcolsep) * \real{0.0857}}
  >{\raggedright\arraybackslash}p{(\columnwidth - 4\tabcolsep) * \real{0.7714}}
  >{\raggedright\arraybackslash}p{(\columnwidth - 4\tabcolsep) * \real{0.1429}}@{}}
\toprule\noalign{}
\begin{minipage}[b]{\linewidth}\raggedright
Key
\end{minipage} & \begin{minipage}[b]{\linewidth}\raggedright
Description and Values
\end{minipage} & \begin{minipage}[b]{\linewidth}\raggedright
Initial Value
\end{minipage} \\
\midrule\noalign{}
\endhead
\bottomrule\noalign{}
\endlastfoot
\texttt{halign} & horizontal alignment: \texttt{l} (left), \texttt{c}
(center), \texttt{r} (right) or \texttt{j} (justify) & \texttt{j} \\
\texttt{valign} & vertical alignment: \texttt{t} (top), \texttt{m}
(middle), \texttt{b} (bottom), \texttt{h} (head) or \texttt{f} (foot) &
\texttt{t} \\
\texttt{wd} & width dimension & None \\
\texttt{bg} & background color name & None \\
\texttt{fg} & foreground color name & None \\
\texttt{font} & font commands & None \\
\texttt{mode} & set cell mode: \texttt{math}, \texttt{imath},
\texttt{dmath} or \texttt{text} & None \\
\texttt{cmd} & execute command for the cell text & None \\
\texttt{preto} & prepend text to the cell & None \\
\texttt{appto} & append text to the cell & None \\
\texttt{r} & number of rows the cell spans & 1 \\
\texttt{c} & number of columns the cell spans & 1 \\
\end{longtable}

Rows:

\begin{longtable}[]{@{}
  >{\raggedright\arraybackslash}p{(\columnwidth - 4\tabcolsep) * \real{0.1071}}
  >{\raggedright\arraybackslash}p{(\columnwidth - 4\tabcolsep) * \real{0.7589}}
  >{\raggedright\arraybackslash}p{(\columnwidth - 4\tabcolsep) * \real{0.1339}}@{}}
\toprule\noalign{}
\begin{minipage}[b]{\linewidth}\raggedright
Key
\end{minipage} & \begin{minipage}[b]{\linewidth}\raggedright
Description and Values
\end{minipage} & \begin{minipage}[b]{\linewidth}\raggedright
Initial Value
\end{minipage} \\
\midrule\noalign{}
\endhead
\bottomrule\noalign{}
\endlastfoot
\texttt{halign} & horizontal alignment: \texttt{l} (left), \texttt{c}
(center), \texttt{r} (right) or \texttt{j} (justify) & \texttt{j} \\
\texttt{valign} & vertical alignment: \texttt{t} (top), \texttt{m}
(middle), \texttt{b} (bottom), \texttt{h} (head) or \texttt{f} (foot) &
\texttt{t} \\
\texttt{ht} & height dimension & None \\
\texttt{bg} & background color name & None \\
\texttt{fg} & foreground color name & None \\
\texttt{font} & font commands & None \\
\texttt{mode} & set mode for row cells: \texttt{math}, \texttt{imath},
\texttt{dmath} or \texttt{text} & None \\
\texttt{cmd} & execute command for every cell text & None \\
\texttt{abovesep} & set vertical space above the row & \texttt{2pt} \\
\texttt{belowsep} & set vertical space below the row & \texttt{2pt} \\
\texttt{rowsep} & set vertical space above and below the row &
\texttt{2pt} \\
\texttt{preto} & prepend text to every cell (like
\texttt{\textgreater{}} specifier in \texttt{rowspec}) & None \\
\texttt{appto} & append text to every cell (like \texttt{\textless{}}
specifier in \texttt{rowspec}) & None \\
\end{longtable}

Columns:

\begin{longtable}[]{@{}
  >{\raggedright\arraybackslash}p{(\columnwidth - 4\tabcolsep) * \real{0.1204}}
  >{\raggedright\arraybackslash}p{(\columnwidth - 4\tabcolsep) * \real{0.7407}}
  >{\raggedright\arraybackslash}p{(\columnwidth - 4\tabcolsep) * \real{0.1389}}@{}}
\toprule\noalign{}
\begin{minipage}[b]{\linewidth}\raggedright
Key
\end{minipage} & \begin{minipage}[b]{\linewidth}\raggedright
Description and Values
\end{minipage} & \begin{minipage}[b]{\linewidth}\raggedright
Initial Value
\end{minipage} \\
\midrule\noalign{}
\endhead
\bottomrule\noalign{}
\endlastfoot
\texttt{halign} & horizontal alignment: \texttt{l} (left), \texttt{c}
(center), \texttt{r} (right) or \texttt{j} (justify) & \texttt{j} \\
\texttt{valign} & vertical alignment: \texttt{t} (top), \texttt{m}
(middle), \texttt{b} (bottom), \texttt{h} (head) or \texttt{f} (foot) &
\texttt{t} \\
\texttt{wd} & width dimension & None \\
\texttt{co} & coefficient for the extendable column (\texttt{X} column)
& None \\
\texttt{bg} & background color name & None \\
\texttt{fg} & foreground color name & None \\
\texttt{font} & font commands & None \\
\texttt{mode} & set mode for column cells: \texttt{math},
\texttt{imath}, \texttt{dmath} or \texttt{text} & None \\
\texttt{cmd} & execute command for every cell text & None \\
\texttt{leftsep} & set horizontal space to the left of the column &
\texttt{6pt} \\
\texttt{rightsep} & set horizontal space to the right of the column &
\texttt{6pt} \\
\texttt{colsep} & set horizontal space to both sides of the column &
\texttt{6pt} \\
\texttt{preto} & prepend text to every cell (like
\texttt{\textgreater{}} specifier in \texttt{colspec}) & None \\
\texttt{appto} & append text to every cell (like \texttt{\textless{}}
specifier in \texttt{colspec}) & None \\
\end{longtable}

hlines:

\begin{longtable}[]{@{}
  >{\raggedright\arraybackslash}p{(\columnwidth - 4\tabcolsep) * \real{0.1398}}
  >{\raggedright\arraybackslash}p{(\columnwidth - 4\tabcolsep) * \real{0.6989}}
  >{\raggedright\arraybackslash}p{(\columnwidth - 4\tabcolsep) * \real{0.1613}}@{}}
\toprule\noalign{}
\begin{minipage}[b]{\linewidth}\raggedright
Key
\end{minipage} & \begin{minipage}[b]{\linewidth}\raggedright
Description and Values
\end{minipage} & \begin{minipage}[b]{\linewidth}\raggedright
Initial Value
\end{minipage} \\
\midrule\noalign{}
\endhead
\bottomrule\noalign{}
\endlastfoot
\texttt{dash} & dash style: \texttt{solid}, \texttt{dashed} or
\texttt{dotted} & \texttt{solid} \\
\texttt{text} & replace hline with text (like \texttt{!} specifier in
\texttt{rowspec}) & None \\
\texttt{wd} & rule width dimension & \texttt{0.4pt} \\
\texttt{fg} & rule color name & None \\
\texttt{leftpos} & crossing or trimming position at the left side &
\texttt{1} \\
\texttt{rightpos} & crossing or trimming position at the right side &
\texttt{1} \\
\texttt{endpos} & adjust leftpos/rightpos for only the
leftmost/rightmost column & \texttt{false} \\
\end{longtable}

vlines:

\begin{longtable}[]{@{}
  >{\raggedright\arraybackslash}p{(\columnwidth - 4\tabcolsep) * \real{0.1333}}
  >{\raggedright\arraybackslash}p{(\columnwidth - 4\tabcolsep) * \real{0.7000}}
  >{\raggedright\arraybackslash}p{(\columnwidth - 4\tabcolsep) * \real{0.1667}}@{}}
\toprule\noalign{}
\begin{minipage}[b]{\linewidth}\raggedright
Key
\end{minipage} & \begin{minipage}[b]{\linewidth}\raggedright
Description and Values
\end{minipage} & \begin{minipage}[b]{\linewidth}\raggedright
Initial Value
\end{minipage} \\
\midrule\noalign{}
\endhead
\bottomrule\noalign{}
\endlastfoot
\texttt{dash} & dash style: \texttt{solid}, \texttt{dashed} or
\texttt{dotted} & \texttt{solid} \\
\texttt{text} & replace vline with text (like \texttt{!} specifier in
\texttt{colspec}) & None \\
\texttt{wd} & rule width dimension & \texttt{0.4pt} \\
\texttt{fg} & rule color name & None \\
\texttt{abovepos} & crossing or trimming position at the above side &
\texttt{0} \\
\texttt{belowpos} & crossing or trimming position at the below side &
\texttt{0} \\
\end{longtable}



\end{document}
